\documentclass[11pt]{book}
\usepackage[usenames,dvipsnames,svgnames,table]{xcolor}
\usepackage[color=yellow]{todonotes}
\hyphenation{pro-fit-ti}
\hyphenation{sub-gra-dien-te}
\hyphenation{pos-sia-mo}

\parindent=0pt \bigskipamount=20pt

\begin{document}

\chapter*{Esercizio 4}

Dati $m$ articoli, ciascuno caratterizzato da un peso $p_j > 0$
($j=1,\dots,m$) ed $n$ veicoli con capacit\`a $a_i>0$ ($i=1,\dots,n$),
si devono determinare gli articoli da caricare sui veicoli in modo
tale che:

\begin{enumerate}
\item il peso totale sul veicolo $i$ non ecceda la capacit\`a $a_i$;
\item ciascun articolo $j$ sia caricato su non pi\`u di un veicolo;
\item il peso globale degli articoli caricati sia massimo;
\item il numero globale degli articoli caricati non sia inferiore ad
  un valore prefissato $k$.
\end{enumerate}

Un possibile modello \`e: 

\begin{center}
\begin{tabular}{lll}
$\max z = \sum\limits_{i=1}^n \sum\limits_{j=1}^m p_j x_{ij}$ & & (1)\\
$\qquad \sum\limits_{j=1}^m p_j x_{ij} \leq a_i$ & $i=1,\dots,n$ & (2) \\
$\qquad \sum\limits_{i=1}^n x_{ij} \leq 1$ & $j=1,\dots,m$ & (3) \\
$\qquad \sum\limits_{i=1}^n\sum\limits_{j=1}^m x_{ij} < k$ & & (4) \\
$\qquad x_{ij} \in \{0, 1 \}$ & $i=1,\dots,n$, $j=1,\dots,m$ & (5) \\
\end{tabular}
\end{center}

\section*{Punto 1}

\textit{Considerando il rilassamento surrogato dei vincoli (2) con
  moltiplicatori tutti pari a 1, determinare un buon upper bound con
  complessit\`a $O(n+m)$ e descrivere la corrispondente procedura
  subgradiente.}

\

Dopo il rilassamento surrogato richiesto il modello \`e:

\begin{center}
\begin{tabular}{lll}
$\max z(\lambda) = \sum\limits_{i=1}^n \sum\limits_{j=1}^m p_j x_{ij}$ & & (1)\\
$\qquad \sum\limits_{i=1}^m\sum\limits_{j=1}^m p_j x_{ij} \leq
\sum\limits_{i=1}^n a_i$ & & (2) \\
$\qquad \sum\limits_{i=1}^n x_{ij} \leq 1$ & $j=1,\dots,m$ & (3) \\
$\qquad \sum\limits_{i=1}^n\sum\limits_{j=1}^m x_{ij} < k$ & & (4) \\
$\qquad x_{ij} \in \{0, 1 \}$ & $i=1,\dots,n$, $j=1,\dots,m$ & (5) \\
\end{tabular}
\end{center}

Poniamo $\sum_{i=1}^n a_i = b$ e questo ha complessit\`a $O(n)$. A
questo punto possiamo porre $y_j = \sum\limits_{i=1}^n x_{ij} $ ed
avere:

\begin{center}
\begin{tabular}{ll}
$\max \sum\limits_{j=1}^m p_jy_j$\\
$\qquad \sum\limits_{j=1}^m p_j y_j \leq b$ \\
$\qquad \sum\limits_{j=1}^m y_j \leq k-1 = \bar{k}$ \\
$\qquad y_j \in \{0,1\}$ & $j=1,\dots,m$
\end{tabular}
\end{center}

Ci\`o che possiamo fare \`e un rilassamento surrogato dei vincoli (con
moltiplicatori $\lambda \geq 0$ e $u \geq 0$ che ci porti ad un
problema KP01. Il nuovo vincolo \`e quindi:

$$
\lambda \sum\limits_{j=1}^m p_jy_j + u \sum\limits_{j=1}^m y_j \leq
\lambda b + u (k -1)
$$

Quindi il nuovo modello \`e:

\begin{center}
\begin{tabular}{ll}
$\max \sum\limits_{j=1}^m p_jy_j$\\
$\qquad \sum\limits_{j=1}^m \tilde{p}_j y_j \leq c$ \\
$\qquad y_j \in \{0,1\}$ & $j=1,\dots,m$
\end{tabular}
\end{center}

Non c'\`e bisogno di preprocessing dato che i coefficienti in funzione
obiettivo sono positivi, pertanto possiamo procedere a fare il
\textbf{rilassamento continuo} e a risolverlo in $O(m)$ con
\textbf{Balas-Zemel}.

\

La complessit\`a totale \`e pertanto $O(n + m)$.

\subsection*{Procedura subgradiente}

I subgradienti dell'ultimo rilassamento sono:

\begin{itemize}
\item $s(\lambda) := b - \sum\limits_{j=1}^n p_j y_j$
\item $s(u) := \bar{k} - \sum\limits_{j=1}^n y_j$
\end{itemize}

Le procedure di aggiornamento dei moltiplicatori sono:

\begin{itemize}
\item $\lambda := \max\{0, \lambda - \sigma s(\lambda)\}$
\item $u := \max\{0, u-\sigma s(u)\}$
\end{itemize}

mentre per aggiornare l'upper bound sar\`a sufficiente $UB := \min \{
UB, \bar{z}\}$.

\section*{Punto 2}

\textit{Determinare un buon upper bound di tipo lagrangiano con
  complessit\`a $O(mn)$ (si consideri $\log m \leq n$) e descrivere la
corrispondente procedura subgradiente}.

\

Rilassiamo in modo lagrangiano (con moltiplicatori $\geq 0$) i vincoli
3 e 4. La funzione obiettivo diventa:

$$
\max \sum\limits_{i=1}^n\sum\limits_{j=1}^m x_{ij} \bigr ( p_j -
p_j\lambda_i - u_j \bigr ) + \sum\limits_{i=1}^n \lambda_i a_i +
\sum\limits_{j=1}^m u_j
$$

Il problema residuo \`e:

\begin{center}
\begin{tabular}{ll}
$\max \sum\limits_{i=1}^n\sum\limits_{j=1}^m \tilde{p}_{ij}x_{ij}$\\
$\qquad\sum\limits_{i=1}^n\sum\limits_{j=1}^m x_{ij} \leq \bar{k}$\\
$\qquad x_{ij} \in \{0,1\}$ & $i=1,\dots,n$, $j=1,\dots,m$ \\
\end{tabular}
\end{center}

I costi lagrangiani sono stati deiniti in $O(nm)$. Notiamo che
abbiamo $n$ problemi KP01 di $m$ variabili (o a scelta $m$ da $n$
variabili), dunque l'UB lo calcoliamo con:

\vspace{20pt}
\begin{tabular}{l}
\textbf{FOR} $i=1,\dots,n$ \textbf{DO}\\
$\qquad$ \textit{Preprocessing} to remove variables with negative costs\\
$\qquad$\textbf{SOLVE} LP relaxation of KP01 on $x_{ij}$ variables
($i$ fixed) with Balas-Zemel\\
\textbf{SUM} every $\bar{z}_i$ to find UB
\end{tabular}
\vspace{20pt}

La complessit\`a ottenuta \`e $O(nm)$.

\end{document}